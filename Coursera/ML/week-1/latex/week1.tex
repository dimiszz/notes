\documentclass[11pt, a4paper]{article}
\usepackage[portuguese]{babel}
\usepackage[margin=1in]{geometry}
\usepackage{amsmath}
\pagenumbering{gobble}


\begin{document}
Objetivo do curso: compreensão básica de modelos de aprendizado de máquina.

\section{O que é Machine Learning?}
É uma área relativamente antiga que é estudada há várias décadas

O que chamou atenção recentemente foram as performances em algumas tarefas interessantes.

Uma delas foi a análise de imagens.

\textbf{Convolutional Neural Network} foi usado em escala pela primeira vez em 2012, o que mostrou sua performance. Desde lá, sua performance apenas melhorou. O \textbf{DeepLearning} está passando a performance humana em alguns casos na análise de imagens.

Outra área que o \textit{Machine Learning} foram os jogos. As maquinas conseguem entender e performar melhor que um humano nos jogos. Um dos melhores jogadores de \textit{Go} perdeu para um humano.\\

Como realmente funciona?

Nós vamos ensinar uma máquina a aprender.

Damos exemplos para ela (amostras de dados), e o que gostaríamos que ela concluísse com os dados.

Para treinar, damos vários exemplos: esse conjunto de dados $x_1$ resulta nesse $y_1$, para $x_2$, $y_2$, e assim por diante.

Para cada $x_i$ que dermos para ele, ele deve retornar o respectivo $y_i$; Porém, se passarmos um conjunto que não foi contemplado, ele deve ser capaz de prever qual será o resultado.

Queremos que ele aprenda os parâmtros do modelo matemático e predizir o que foi pedido.

\subsection{Regressão Lógica}

Queremos um training set que consiga aprender um modelo e consegue prever o resultado dado um conjunto de dados.

Para fazer o \textit{learning} temos um algorítimo que é feito com vários parâmetros e \textit{learning} significa que gostaríamos de inferir quais parâmetros desse modelo são consistentes com nossos dados de treinamento.

Vamos considerar um dos algorítmos mais básicos: \textbf{Logistic Regression}

O objetivo do \textbf{Machine Learning} é que, dado $N$ exemplos, data $x$ e outcome $y$, gostaríamos de construir modelo preditivo que é capaz de prever $y$ dado $x$.\\

\textbf{Linear Predictive Model}: $X_{i1}$ é o primeiro componente do vetor $X$, $X_{i2}$ o segundo e assim por diante até $X_{iM}$.

Vamos multiplicar cada componente do vetor $X$ por um parâmtro e somamos um bias:
$$(b_1 \times x_{i1}) + (b_2 \times x_{i2}) + \dots + (b_M \times x_{iM}) + b_0 $$

Isso é um mapeamento dos dados $X_i$ para um número $Z_i$.

Muitas vezes é melhor dar uma chance se vai chover ou não em um dia do que afirmar algo. Para fazer isso, usamos uma \textbf{Logistic Function} notado por $\sigma$:
$$ p(y_i = 1 | x_i) = \sigma (z_i) $$
$z_i$: multiplicação dos parâmetros dos dados $X$ com os parâmtros $b_1, b_2$ até $b_M$

Essa função, a \textbf{Sigmoid Function} $p(y_i = 1 | x_i) = \sigma(z_i)$, sempre está entre 0 e 1. Quando $z_i$ é grande, como 5 ou 6, a função converte ele para um número perto de 1. Quando é pequeno, -1, -2, -4, converte para perto de 0.


\textbf{Sigmoid Function} é uma maneira de converter previsões para uma perspectiva probabilística.


Os parâmetros $b$ dizem o quão importante as variáveis são para a predição.


É um modelo bem simples; é apenas uma combinação linear de multiplicação das variáveis observados pelos parâmetros associados, somando-os, mapenado eles para uma variável $z_i$ e, então, executando-os por meio de uma função Sigmoid Function.


O coração do machine learning é: temos um modelo paramétrico que é caracterizado por um conjunto de parâmetros que queremos aprender. A maneira em que fazemos o aprendizado é ter um conjunto de dados e, para esses dados, temos parâmetros $X$ e resultado $Y$. Gostaríamos de aprender os parâmetros do nosso modelo de tal forma que as predições do modelo sejam consistentes com os dados do treinamento.

O que queremos dizer com \textit{Learning} é inferir os parâmetros $B_0$ até $B_M$ que nos forneçam saídas de mapeamento de $X$ para $Y$ consistente com os dados.

Os conceitos básicos da \textbf{Logistic Regression} são bastante usados em Deep Learning.

\textbf{Logistic Regression} é um processo de modelar uma probabilidade discreta de um resultado dado os inputs. O mais comum é um binário, sim ou não.
\end{document}