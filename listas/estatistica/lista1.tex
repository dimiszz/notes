\documentclass[11pt, a4paper]{article}
\usepackage[portuguese]{babel}
\usepackage[margin=1in]{geometry}
\usepackage{amsmath}
\pagenumbering{gobble}
\begin{document}
\title{Lista Estatística}
\date{}
\maketitle
\textbf{Exercícios}.
Para responder, é interessante dar exemplos.
\begin{enumerate}
\item O que é estatística?

\item O que é experimento? Quais suas limitações?
\item O que é evento? Comente sobre 3 de suas propriedades.
\item Defina espaço amostral e a sua diferença entre espaço mensurável.
\item O que é probabilidade? Comente sobre 5 propriedades.
\item Quando podemos usar técnicas de contagem? Dê um exemplo prático com moedas.
\item No que se baseia a probabilidade condicional? Faça um diagrama de Venn para representar isso.
\item Como podemos expressar $Pr(A\cap B)$ usando a probabilidade condicional?
\item Se condicionásse-mos o item anterior em C, $Pr(A\cap B\,|\,C)$, como seria o resultado? Explique.
\item Como podemos expandir $Pr(A)$ em subconjuntos?

\item O que são eventos independentes? Explique e faça um exemplo usando um diagrama de Venn.
\item Para cada item, Diga se é um evento Independente:
	\begin{enumerate}
	\item Sortear 2 cartas de baralho.
	\item Lançar uma moeda.
	\item Sortear 2 cartas de 2 baralhos diferentes.
	\item Lançar um dado $n$ vezes.
	\item O mercado financeiro
	\end{enumerate}
\item Qual a probabilidade dessa situação? 4 cartas são retiradas de um baralho, uma por vez. A primeira é vermelha, a segunda é preta, a terceira de paus e a quarta é um 5.
\item O que são eventos condicionalmente independentes?
\item Dê outra definição para $Pr(A\,|\,B)$.
\item O que é uma Variável Aleatória?
\item Fale qual distribuição encaixa em cada evento e explique o porque:
	\begin{enumerate}
	\item retirar $n$ cartas, sem reposição. 
	\\V.A.: Quantidade de cartas de copas.
	
	\item  Retirar uma carta do baralho. 
	\\V.A.: 1 se a carta é A, 2 se a carta é um 2, até que 13 se a carta é um K.
	\item Retire uma carta do baralho. 
	\\V.A.: 1 se a carta é de copas, 0 caso contrário.
	
	\end{enumerate}
\end{enumerate}

\bibliographystyle{elsarticle-num}
\bibliography{mybibfile}

\end{document}