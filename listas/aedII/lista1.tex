\documentclass[11pt, a4paper]{article}
\usepackage[portuguese]{babel}
\usepackage[margin=1in]{geometry}
\usepackage{amsmath}
\pagenumbering{gobble}

\begin{document}
	\begin{center}
	\textbf{Revisão Algoritmos e Estruturas de Dados II}
	\end{center}
\textbf{Exercícios}.
Para responder, é interessante dar exemplos.

\begin{enumerate}
\item O que é um grafo? Como é representado?
\item Qual a utilidade de um grafo?
\item Como é a representação de uma aresta?
\item Comente as diferenças entre grafo dirigido e não dirigidos (não direcionados). Como funciona a relação de adjancência entre eles?
\item Explique os seguintes conceitos sobre grafos e de exemplos:
	\begin{enumerate}
	\item Grau de saída e entrada;
	\item Caminho e comprimento;
	\item Ciclo;
	\item Grafos conexos;
	\item Grafo fortemente conexo;
	\item Grafo fracamente conexo;
	\item Um Clique;
	\item Grafos ponderados;
	\item Grafos transpostos;
	\item Subgrafos;
	\item Subgrafo gerador;
	\item Subgrafo induzido;
	\item Subgrafo próprio;
	\item Grafo bipartido;
	\item Multigrafo;
	\item Grafo Misto;
	\end{enumerate}
\item Como podemos representar grafos dirigidos e ponderados? Quando usar cada representação?


\end{enumerate}
\end{document}
