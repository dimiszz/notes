\documentclass[11pt, a4paper]{article}
\usepackage[portuguese]{babel}
\usepackage[margin=1in]{geometry}
\usepackage{amsmath}
\pagenumbering{gobble}
\usepackage{listings}
\usepackage{graphicx}
\usepackage{wrapfig}

\begin{document}

\begin{enumerate}

\item Suponha que $X_1, \dots , X_n$ formem uma amostra aleatória de uma distribuição com a seguinte p.d.f:

\[ 
f(x;\theta) = 
\begin{cases}\frac{\theta-1}{x^{\theta}} &\, \text{para } x \geq 1\\
0 &\, \text{caso contrário}
\end{cases}
\]
Também, suponha que o valor de $\theta > 1$ é desconhecido.

\begin{enumerate}
 
\item Determine o M.L.E. de $\theta$.
\vspace{4cm}
\item Suponha uma outra variável aleatória $Y = \log \, X $ e foi observada a seguinte amostra: 0.2; 0.5; e 0.7. Utilize o estimador do item (a) e determine o valor de $\theta$.
\vspace{4cm}
\end{enumerate}

\item Considere uma ÚNICA amostra X da seguinte distribuição:
\[ 
f(x;\theta) = 
\begin{cases}\frac{2\theta x + 1}{\theta + 1} &\, \text{para } 0 < x < 1\\
0 &\, \text{caso contrário}
\end{cases}
\]
Onde $\theta \in [0, 10]$ é desconhecido.

\begin{enumerate}
\item Calcule o M.L.E de $\theta$.

\vspace{4cm}

\item Calcule o estimador de $\theta$ usando o Método dos Momentos.
\vspace{4cm}

\end{enumerate}

\end{enumerate}
\end{document}